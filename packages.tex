%  Encoding: UTF-8
%
% %%%%%%%%%%%%%%%%%%%%%%%%%%%%%%%%%%%%%%%%%%%%%%%%%%%%%%%%%%%%%%%%%%%%%%%%%%%% %
%  Copyright (c) 2021 I.F.F. dos SANTOS
%
%  Permission is hereby granted, free of charge, to any person obtaining a copy 
%  of this software and associated documentation files (the “Software”), to 
%  deal in the Software without restriction, including without limitation the 
%  rights to use, copy, modify, merge, publish, distribute, sublicense, and/or
%  sell copies of the Software, and to permit persons to whom the Software is 
%  furnished to do so, subject to the following conditions:
%
%  The above copyright notice and this permission notice shall be included in 
%  all copies or substantial portions of the Software.
%
%  THE SOFTWARE IS PROVIDED “AS IS”, WITHOUT WARRANTY OF ANY KIND, EXPRESS OR 
%  IMPLIED, INCLUDING BUT NOT LIMITED TO THE WARRANTIES OF MERCHANTABILITY, 
%  FITNESS FOR A PARTICULAR PURPOSE AND NONINFRINGEMENT. IN NO EVENT SHALL THE 
%  AUTHORS OR COPYRIGHT HOLDERS BE LIABLE FOR ANY CLAIM, DAMAGES OR OTHER 
%  LIABILITY, WHETHER IN AN ACTION OF CONTRACT, TORT OR OTHERWISE, ARISING 
%  FROM, OUT OF OR IN CONNECTION WITH THE SOFTWARE OR THE USE OR OTHER DEALINGS 
%  IN THE SOFTWARE.
% %%%%%%%%%%%%%%%%%%%%%%%%%%%%%%%%%%%%%%%%%%%%%%%%%%%%%%%%%%%%%%%%%%%%%%%%%%%% %


% --------------------------------------
% Aparência
% --------------------------------------

\usepackage[T1]{fontenc}               % Compilar correctamente com pdflatex
\usepackage[utf8x]{inputenc}           % Codificacao do documento (conversão automática dos acentos).
\usepackage{ucs}                       % Complemento do anterior.
\usepackage{indentfirst}               % Indenta o primeiro parágrafo

\PassOptionsToPackage{
   english,             % idioma adicional para hifenização.
   french,              % idioma adicional para hifenização.
   spanish,             % idioma adicional para hifenização.
   main=portuguese      % Idioma do documento
}{babel}

% 1ex - É a largura da letra x (minúscula)
% 1em - É a largura da letra M (Maiúscula)
\setlength{\parindent}{3ex}  % Tamanho 
\setlength{\parskip}{1em}    % Tamaño del parágrafo

% Margens da página
\usepackage{geometry}                          % Permite configurar as margens da página.
\geometry{hmargin={3cm,2cm},vmargin={3cm,2cm}} % Margens conforme padrão ABNT

% Pacote de citações
\usepackage[num,overcite]{abntex2cite} % Citações padrão ABNT, em ordem alfabética.
\citebrackets[]                        % Citações com coxetes.
\usepackage{bibentry}                  %

% Usar font 'TeX Gyre Termes'
\usepackage{tgtermes}

\usepackage{titlesec,titletoc}

\titleformat{\chapter}[hang]{\large\bfseries}{\thechapter. }{0pt}{\large\bfseries}
\titleformat{\section}[hang]{\normalfont\bfseries}{\thesection. }{0pt}{\normalfont\bfseries}

% --------------------------------------
% ETC
% --------------------------------------

\usepackage{graphicx}                  % Inclusão de figuras.
\usepackage{wrapfig}                   % Figuras junto com o texto
\usepackage{float}                     % Para posicionar figuras corretamente.
\usepackage{subcaption}                % Para legendas nas subfiguras.

\usepackage{amssymb}                   % Para exibir símbolos de conjuntos de números (reais, etc...).
\usepackage{amsmath}                   % Para adcionar equações
\usepackage{amsfonts}                  % Fontes para notação matemática. de cada seção.
\usepackage{esint}                     % various fancy integral symbols


\usepackage{microtype}                 % para melhorias de justificação.
\usepackage{fancyhdr}                  % Pemite alterações no cabeçalho e rodapé.
\usepackage{hyperref}                  % Cria formatação automática de PDF.
\usepackage[hyphenbreaks]{breakurl}    % Quebra de linha em url.
\usepackage{lettrine}				      % letras capitulares

\usepackage[htt]{hyphenat}  % Permitir hifenização usando \texttt

\usepackage{enumitem}

% --------------------------------------
% Para notación formal de proposiciónes
% --------------------------------------
\usepackage{amsthm, thmtools}

\declaretheoremstyle[
   headfont=\normalfont\bfseries,
   bodyfont=\normalfont\itshape,
   notefont=\normalfont\bfseries,
   notebraces={ (}{)},
   headpunct={. },
   numbered=yes,
   spaceabove=1em,
   postheadspace=0em,
   spacebelow=1em,
   qed={ }
]{EstiloItalico}
\declaretheoremstyle[
   headfont=\normalfont\bfseries,
   bodyfont=\normalfont\mdseries,
   notefont=\normalfont\bfseries,
   notebraces={ (}{)},
   headpunct={. },
   numbered=yes,
   spaceabove=1em,
   postheadspace=0em,
   spacebelow=1em,
   qed={ }
]{EstiloPlano}

% Lei
\declaretheorem[
   style=EstiloItalico,
   parent=chapter,
   name=Lei
]{lei}

% Axiomas
\declaretheorem[
   style=EstiloItalico,
   parent=chapter,
   name=Axioma
]{axioma}

% Postulados
\declaretheorem[
   style=EstiloItalico,
   parent=chapter,
   name=Postulado
]{postulado}

% Definição
\declaretheorem[
   style=EstiloPlano,
   parent=chapter,
   name=Defini\c{c}\~ao
]{definicao}

% Teorema
\declaretheorem[
   style=EstiloItalico,
   parent=chapter,
   name=Teorema
]{teorema}

% Prova
\declaretheorem[
   style=EstiloPlano,
   numbered=no,
   qed={C.Q.D.},
   name=Desmostra\c{c}\~ao
]{prova}

% Exemplo
\declaretheorem[
   style=EstiloPlano,
   parent=chapter,
   name=Exemplo
]{exemplo}

% Pergunta
\declaretheorem[
   style=EstiloItalico,
   name=Pergunta
]{pergunta}

% --------------------------------------
% Para exibição de código fonte
% --------------------------------------

% see https://en.wikibooks.org/wiki/LaTeX/Source_Code_Listings

\usepackage{listings}
\usepackage{xcolor}

% see https://latexcolor.com/
\definecolor{codegreen}{rgb}{0,0.6,0}
\definecolor{codegray}{rgb}{0.5,0.5,0.5}
\definecolor{amaranth}{rgb}{0.9, 0.17, 0.31}

\usepackage{caption}
\DeclareCaptionFont{white}{\color{white}}
\DeclareCaptionFormat{listing}{\hspace*{-0.4pt}\colorbox{gray}{\parbox{\textwidth}{#1#2#3}}}
\captionsetup[lstlisting]{format=listing,labelfont=white,textfont=white}

\renewcommand{\lstlistingname}{Código}

\lstset{
   backgroundcolor=\color{white},   % choose the background color; you must add \usepackage{color} or \usepackage{xcolor}; should come as last argument
   basicstyle=\mdseries\ttfamily\scriptsize, % the size of the fonts that are used for the code
   breakatwhitespace=false,         % sets if automatic breaks should only happen at whitespace
   breaklines=true,                 % sets automatic line breaking
   captionpos=t,                    % sets the caption-position to bottom
   columns=fixed,                   % Using fixed column width (for e.g. nice alignment)
   escapechar=µ,
   numbers=none,                    % Not use numbers
   frame=single,	                  % adds a frame around the code
   keepspaces=true,                 % keeps spaces in text, useful for keeping indentation of code (possibly needs columns=flexible)
   rulecolor=\color{black},         % if not set, the frame-color may be changed on line-breaks within not-black text (e.g. comments (green here))
   showspaces=false,                % show spaces everywhere adding particular underscores; it overrides 'showstringspaces'
   showstringspaces=false,          % underline spaces within strings only
   showtabs=false,                  % show tabs within strings adding particular underscores
   tabsize=3  	                     % sets default tabsize to 3 spaces
}

\lstdefinestyle{c}{
   language=C, % the language of the code (can be overrided per snippet)
   commentstyle=\color{codegray}, % comment style
   keywordstyle={\color{codegreen}\bfseries},
   stringstyle=\color{amaranth} % string literal style
}

\lstdefinestyle{f90}{
   language=Fortran, % the language of the code (can be overrided per snippet)
   commentstyle=\color{codegray}, % comment style
   keywordstyle={\color{codegreen}\bfseries},
   stringstyle=\color{amaranth} % string literal style
}

\lstdefinestyle{java}{
   language=Java, % the language of the code (can be overrided per snippet)
   commentstyle=\color{codegray}, % comment style
   keywordstyle={\color{codegreen}\bfseries},
   stringstyle=\color{amaranth} % string literal style
}

\lstdefinestyle{py}{
   language=Python, % the language of the code (can be overrided per snippet)
   commentstyle=\color{codegray}, % comment style
   keywordstyle={\color{codegreen}\bfseries},
   stringstyle=\color{amaranth} % string literal style
}

\lstdefinestyle{gnuplot}{
   language=Gnuplot, % the language of the code (can be overrided per snippet)
   commentstyle=\color{codegray}, % comment style
   keywordstyle={\color{codegreen}\bfseries},
   stringstyle=\color{amaranth} % string literal style
}

\lstdefinestyle{bash}{
   language=bash, % the language of the code (can be overrided per snippet)
   commentstyle=\color{codegray}, % comment style
   keywordstyle={\color{codegreen}\bfseries},
   stringstyle=\color{amaranth}, % string literal style
   morekeywords={apt, pacman, yum, zypper, dnf, dns, mkdir, cp, configure, make, tar},
}

\lstdefinestyle{pseudo}{
   commentstyle=\color{codegray}, % comment style
   keywordstyle={\color{codegreen}},
   stringstyle=\color{amaranth}, % string literal style
   morekeywords={inicio, fim, para, ate, subrotinas, se, senao, entao, funcao, inteiro, enquanto, gnuplot, logico, verdadeiro, subrotina, variavel},
   morestring=[b]",
   morecomment={[l]//},
   inputencoding=utf8,
   extendedchars=true,
   literate=%
      {á}{{\'a}}1 {é}{{\'e}}1 {í}{{\'i}}1 {ó}{{\'o}}1 {ú}{{\'u}}1
      {Á}{{\'A}}1 {É}{{\'E}}1 {Í}{{\'I}}1 {Ó}{{\'O}}1 {Ú}{{\'U}}1
      {à}{{\`a}}1 {è}{{\`e}}1 {ì}{{\`i}}1 {ò}{{\`o}}1 {ù}{{\`u}}1
      {À}{{\`A}}1 {È}{{\'E}}1 {Ì}{{\`I}}1 {Ò}{{\`O}}1 {Ù}{{\`U}}1
      {ä}{{\"a}}1 {ë}{{\"e}}1 {ï}{{\"i}}1 {ö}{{\"o}}1 {ü}{{\"u}}1
      {Ä}{{\"A}}1 {Ë}{{\"E}}1 {Ï}{{\"I}}1 {Ö}{{\"O}}1 {Ü}{{\"U}}1
      {â}{{\^a}}1 {ê}{{\^e}}1 {î}{{\^i}}1 {ô}{{\^o}}1 {û}{{\^u}}1
      {Â}{{\^A}}1 {Ê}{{\^E}}1 {Î}{{\^I}}1 {Ô}{{\^O}}1 {Û}{{\^U}}1
      {ã}{{\~a}}1 {ẽ}{{\~e}}1 {ĩ}{{\~i}}1 {õ}{{\~o}}1 {ũ}{{\~u}}1
      {Ã}{{\~A}}1 {Ẽ}{{\~E}}1 {Ĩ}{{\~I}}1 {Õ}{{\~O}}1 {Ũ}{{\~U}}1
      {œ}{{\oe}}1 {Œ}{{\OE}}1 {æ}{{\ae}}1 {Æ}{{\AE}}1 {ß}{{\ss}}1
      {ű}{{\H{u}}}1 {Ű}{{\H{U}}}1 {ő}{{\H{o}}}1 {Ő}{{\H{O}}}1
      {ç}{{\c c}}1 {Ç}{{\c C}}1 {ø}{{\o}}1 {å}{{\r a}}1 {Å}{{\r A}}1
      {€}{{\euro}}1 {£}{{\pounds}}1 {«}{{\guillemotleft}}1
      {»}{{\guillemotright}}1 {ñ}{{\~n}}1 {Ñ}{{\~N}}1 {¿}{{?`}}1 {¡}{{!`}}1
      {←}{{$\leftarrow$}}1
      {!=}{{$\neq$}}1
}

% --------------------------------------
% Para fluxogramas
% --------------------------------------

\usepackage{tikz}

\usetikzlibrary{shapes.geometric, arrows}
\tikzstyle{startstop} = [rectangle, rounded corners, minimum width=3cm, minimum height=1cm,text centered, draw=black]
\tikzstyle{io} = [trapezium, trapezium left angle=70, trapezium right angle=110, minimum width=3cm, minimum height=1cm, text centered, draw=black]
\tikzstyle{process} = [rectangle, minimum width=3cm, minimum height=1cm, text centered, draw=black]
\tikzstyle{decision} = [diamond, minimum width=3cm, minimum height=1cm, text centered, draw=black]
\tikzstyle{conector} = [circle, minimum width=0.5cm, minimum height=0.5cm, text centered, draw=black]
\tikzstyle{arrow} = [thick,->,>=stealth]
\tikzstyle{line} = [draw, -latex']

% --------------------------------------
% CONFIGURAÇÕES DE PACOTES
% --------------------------------------
\newcommand{\cqd}{C.Q.D.}

\newcommand{\R}{\mathbb{R}}
\newcommand{\C}{\mathbb{C}}
\newcommand{\Z}{\mathbb{Z}}
\newcommand{\N}{\mathbb{N}}
\newcommand{\E}{\mathbb{E}}
\newcommand{\M}{\mathbb{M}}

\newcommand{\F}{\mathcal{F}}

\newcommand{\sen}{\operatorname{sen}}
\newcommand{\asen}{\,\operatorname{arcsen}\,}

\newcommand{\bra}[1]{\langle#1|}
\newcommand{\ket}[1]{|#1\rangle}
\newcommand{\braket}[2]{\langle#1|#2\rangle}
\newcommand{\ketbra}[2]{|#1\rangle\langle#2|}

\newcommand{\eq}[1]{\hyperref[eq:#1]{equa\c{c}\~ao} \ref{eq:#1}}

\newcommand{\aspas}[1]{``#1''}

\newcommand{\gnuplot}{\ttfamily gnuplot \rmfamily }
