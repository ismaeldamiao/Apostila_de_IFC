% %%%%%%%%%%%%%%%%%%%%%%%%%%%%%%%%%%%%%%%%%%%%%%%%%%%%%%%%%%%%%%%%%%%%%%%%%%%%%%
% Linux
% %%%%%%%%%%%%%%%%%%%%%%%%%%%%%%%%%%%%%%%%%%%%%%%%%%%%%%%%%%%%%%%%%%%%%%%%%%%%%%
\chapter{Instalando C, Fortran, Java e/ou Python no Linux}

\lettrine{P}{ara} instalar os compiladores de C, Fortran e Java, bem como a JVM
e o intérprete de Python
no Debian e derivados, como Ubuntu, Linux Mint, etc basta digitar
o comando abaixo no terminal.
\begin{lstlisting}[style=bash]
apt update && apt install -y gcc gfortran openjdk-17-jdk python3
\end{lstlisting}

% %%%%%%%%%%%%%%%%%%%%%%%%%%%%%%%%%%%%%%%%%%%%%%%%%%%%%%%%%%%%%%%%%%%%%%%%%%%%%%
% Android
% %%%%%%%%%%%%%%%%%%%%%%%%%%%%%%%%%%%%%%%%%%%%%%%%%%%%%%%%%%%%%%%%%%%%%%%%%%%%%%
\chapter{Instalando C, Fortran, Java e/ou Python no Android}

\lettrine{A}{} primeira coisa que vamos precisar para programar pelo
telefone é de um terminal e o melhor que há é o Termux
que pode ser baixado em
\url{https://f-droid.org/packages/com.termux/}. Após baixar e instalar o Termux
abra ele e digite as seguintes linhas de comando (tecle enter depois de cada uma).
\begin{lstlisting}[style=bash]
apt update
apt install -y wget gnupg
wget -O - cctools.info/public.key | apt-key add -
echo "deb https://cctools.info termux cctools" >\
$PREFIX/etc/apt/sources.list.d/cctools.list
\end{lstlisting}

Finalmente, para
instalar os compiladores de C, Fortran e Java, bem como a JVM
e o intérprete de Python
no Android basta digitar
o comando abaixo.
\begin{lstlisting}[style=bash]
apt update && apt install -y gcc-cctools openjdk-17 python
\end{lstlisting}

Para facilitar o uso do \texttt{gfortran} digite o
seguinte comando no terminal.
\begin{lstlisting}[style=bash]
ln -s ~/../cctools-toolchain/bin/gfortran $PREFIX/bin/gfortran
\end{lstlisting}


% %%%%%%%%%%%%%%%%%%%%%%%%%%%%%%%%%%%%%%%%%%%%%%%%%%%%%%%%%%%%%%%%%%%%%%%%%%%%%%
% Windows
% %%%%%%%%%%%%%%%%%%%%%%%%%%%%%%%%%%%%%%%%%%%%%%%%%%%%%%%%%%%%%%%%%%%%%%%%%%%%%%
\chapter{Instalando C, Fortran, Java e/ou Python no Windows}

\lettrine{B}{aixe} o instalador do OpenJDK (implementação gratis e de código aberto do java)
em
\url{https://docs.microsoft.com/pt-br/java/openjdk/download}
(procure pelo arquivo de extensão \texttt{msi})
e execute ele, certifique-se de marcar a opção \textit{Add to PATH} na
janela \textit{Custom Setup}.

Baixe o instalador do Python em
\url{https://www.python.org/downloads/release/python-3100/}
e execute ele, lembre-se de marcar a opção \textit{Add Python 3.10 to PATH}
antes de começar a instalação.
Na última janela pode aparecer a opção \textit{Disable path length limit},
execute-a.

Baixe o MinGW (Pacote que possui o GCC, que possui compiladores de C e Fortran)
em
\url{https://sourceforge.net/projects/mingw/}
e execute ele,
quando aparecer a janela \textit{MinGW Instalation Manager}
clique em \textit{Basic Setup} (à esquerda) e depois selecione, ao lado,
os pacotes \texttt{mingw32-base} e \texttt{mingw32-gcc-gfortran}
para serem instalados --- para fazer isso basta clicar no quadradinho e
depois em \textit{Mark for Installation} ---.
Depois vá à aba \textit{Installation} (em cima) e clique em \textit{Apply Changes}.
Uma vez concluida a instalação é preciso adicionar o MinGW ao PATH,
para isso pesquise no menu iniciar por
\texttt{exibir configurações avançadas do sistema} e clique na opção
\texttt{Variáveis de Ambiente}, procure pela variável \texttt{Path}
e clique em \texttt{editar} e depois em \texttt{Novo}
e escreva \aspas{\texttt{C:{\textbackslash}MinGW{\textbackslash}bin}}.


% %%%%%%%%%%%%%%%%%%%%%%%%%%%%%%%%%%%%%%%%%%%%%%%%%%%%%%%%%%%%%%%%%%%%%%%%%%%%%%
% Editores de texto e IDE's
% %%%%%%%%%%%%%%%%%%%%%%%%%%%%%%%%%%%%%%%%%%%%%%%%%%%%%%%%%%%%%%%%%%%%%%%%%%%%%%
\chapter{Editores de texto e IDE's}

Para escrever códigos precisamos de editores de texto, em geral
queremos editores que deixam as palavras coloridas conforme sua utilidade.

Linux:
\begin{itemize}[nosep]
\item Xed \url{https://github.com/linuxmint/xed}
\item Gedit \url{https://wiki.gnome.org/Apps/Gedit}
\end{itemize}

Android:
\begin{itemize}[nosep]
\item DroidEdit \url{https://play.google.com/store/apps/details?id=com.aor.droidedit}
\end{itemize}

Windows:
\begin{itemize}[nosep]
\item Gedit \url{https://wiki.gnome.org/Apps/Gedit}
\end{itemize}

Uma IDE nada mais é que um editor de texto que possui várias outras funções
para facilitar a escrita de códigos, em geral uma IDE é dedicada a uma
linguagem específica.

Linux:
\begin{itemize}[nosep]
\item Eclipse \url{https://www.eclipse.org/downloads/}
\item Visual Studio Code \url{https://code.visualstudio.com/download}
\item Code::Blocks (C, C++, Fortran) \url{https://www.codeblocks.org/downloads/binaries/}
\item Apache NetBeans \url{https://netbeans.apache.org/download/index.html}
\item Anjuta \url{https://anjuta.org/}
\end{itemize}

Android:
\begin{itemize}[nosep]
\item AIDE (C, Java) \url{https://play.google.com/store/apps/details?id=com.aide.ui}
\end{itemize}

Windows:
\begin{itemize}[nosep]
\item Eclipse \url{https://www.eclipse.org/downloads/}
\item Visual Studio Code \url{https://code.visualstudio.com/download}
\item Code::Blocks (C, C++, Fortran) \url{https://www.codeblocks.org/downloads/binaries/}
\item Apache NetBeans \url{https://netbeans.apache.org/download/index.html}
\end{itemize}

